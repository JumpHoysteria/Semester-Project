\section{Theory}

\subsection{Different strategies in making conductive, biocompatible fibers.}



\subsection{Polyurethane}

Polyurethanes (PUR) are biocompatible and biostable polymers, with urethane as the characteristic group. Due to most PUR being classically synthesized using polycondensation of diisocyanates with alcohols and amines, two side groups per monomer are introduced. This leaves the option for functionalization, depending on the requirement of the application. 

On one hand, they're moldable and have favourable tensile and fatigue properties, all of which makes them a suitable choice of material for the development of most complex biomedical devices. Until now, PUR has been extensively studied in the context of small vascular shunts and cardiac assist devices, where the thromboresistant property of the PUR fit formidable. Under most physiologic conditions, PUR are degradation resistant and can handle stresses very well. [Ulery] 
On the other hand, PUR's main usages are far more mundane. Due to being adhesive it is mainly used as grounding for painting. During everyday life, one may encounter PUR as usual kitchen sponges, filling in car seats and spandex.

However, PUR, is not electrically conductive. This makes it impossible to introduce electrical functionalities with pure PUR implants, like in-device monitoring, wireless implant-environment-communication or resistivity-coupled strain sensing.



\subsection{Gold Salt}

When evaluating, which conductive material to use, noble metals seem to host quite some promising candidates. The two main properties we wanted to base our decision on were biocompatibility and conductivity. Silver is highly conductive, but also severly cytotoxic. Platinum, on the other hand, is not conductive, but biocompatible, although there's a growing body of evidence that proves otherwise. As a consequence we decided on using gold, as it strikes the balance between conductivity and biocompatibility expertly. Also there is robust evidence, that gold stays biocompatible, even as gold nanoparticles (AuNP). This is especially important, when considering that this is the relevant size range for this work. [Liu][Shukla]

Due to practical reasons, we had an interest in increasing the efficiency. This is typically done by minimizing the amount of substance that is wasted in non-targeted interactions. Gold in bulk is highly inert and therefore difficult to incorporate in a directed chemical approach, whereas the ionic form is a well studied agent in Redox-reactions (i.e.the method that was chosen in this work). The increase in efficiency can be achieved using a spatially selective approach with gold salt, where only the fiber is decorated with gold.

Gold(III) chloride trihydrate was therefore the reagent of choice.

\subsection{Reduction agent}

Redox-reactions are electrochemical reactions where electrons are exchanged between the participating agents. Typically the reaction is described by using two so-called half-reactions.The reduction half-reaction describes the process of loosing, whereas the oxidation counterpart describes gaining electrons, respectively. As one might expect, the presence of a specific half reaction to happen and the direction in which it happens (Oxidation or Reduction), depends on its inherent likelihood of happening and its relation to the likelihood of the present alternatives of half-reactions. In the Redox-regime, the likelihood is coupled to a so-called standard-reduction potential (\textit{E\textsuperscript{0}}, [V]). The more negative the SRP of a half-reaction, the more likely the reaction to happen. Every half reaction is reversible and will be reversed when a half reaction with a lower \textit{E\textsuperscript{0}} is encountered. \\[0.4cm]

 \begin{center}
 
\schemestart 
\ce{[AuCl4]-} + 3$\mathrm{e^-}$  \arrow{->} \ce{Au{(s)}} + 4\ce{Cl-}, E\textsuperscript{0}\textsubscript{Gold} = +0.93 V 
\schemestop\par 
\label{Scheme:Generic}

 \end{center}
 \begin{center}
 \schemestart 
3Red \arrow{->} 3$\mathrm{Red^+}$ + 3$\mathrm{e^-}$, E\textsuperscript{0}\textsubscript{Red} $\mathrm{>}$ +0.93 V
\schemestop\par %
 \end{center}


The desired reaction to happen, is the reduction of the gold-chloride to get gold in its solid form (denoted by \ce{Au{(s)}}). Gold in bulk is electrically conductive and is highly bioinert.

When choosing potential candidates, we defined several qualities, which the desired reduction agent should have.

First, it should reduce gold efficiently and exclusively. On one hand, this confines the list of potentially chosen agents to those who comply with E\textsuperscript{0}\textsubscript{Red} $\mathrm{>}$ +0.93 V, according to Redox-Theory. On the other hand we want to minimise interactions between the reduction agent and PUR, due to possibly modified mechanical, chemical and biocompatible properties. This happens more often with very aggressive reduction agents, which have been shown to introduce radicals and modification of the surface properties. Hence, we want to avoid very strong reaction agents. 

Second, we want it to biocompatible. Since the goal of this work lies in applying knowledge in biomedical applications, biocompatibility is outstandingly important. If the reduction agents itself fulfills this requirement, then there is no need for introducing another washing-step, which acts an another degree of freedom with multiple failure modes.


Third, the reduction agent should be easily accessible and not too expensive, to facilitate entry to technology and faster iterations when developing applications.


