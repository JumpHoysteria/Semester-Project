\section{Theory}

\subsection{Polyurethane}

Polyurethanes (PUR) are biocompatible and biostable polymers, with urethane as the characteristic group. Due to most PUR being classically synthesized using polycondensation of diisocyanates with alcohols and amines, two side groups are introduced. This leaves the option for functionalization, depending on the requirement of the application. 

On one hand, they're moldable and have favourable tensile and fatigue properties, all of which makes them a suitable choice of material for the development of most complex biomedical devices. Until now, PUR has been extensively studied in the context of small vascular shunts and cardiac assist devices, where the thromboresistant property of the PUR fit formidable. Under most physiologic conditions, PUR are degradation resistant and can handle stresses very well. [Ulery] 
On the other hand, PUR's main usages are far more mundane. Due to being adhesive it is mainly used as grounding for painting. During everyday life, one may encounter PUR as usual kitchen sponges, filling in car seats and spandex.

However, PUR, is not electrically conductive. This makes it impossible to introduce electrical functionalities with pure PUR implants, like in-device monitoring, wireless implant-environment-communication or conductivity-coupled strain sensing.


\subsection{Gold Salt}

When having to decide, which conductive material to use, noble metals seem to host quite some promising candidates. The two main properties we wanted to base our decision on are biocompatibility and conductivity. Silver is highly conductive, but also severly cytotoxic. Platinum, on the other hand, is not conductive, but biocompatible, even though there's a growing body of evidence that proves otherwise. As a consequence we decided on using gold, as it provides the best compromise between conductivity and biocompatibility. Also there is robust evidence, that gold stays biocompatible, even as gold nanoparticles (AuNP). This is especially important, when considering that this is the relevant size range for this work. [Liu][Shukla]

Due to fiscal reasons, we had an interest in increasing the efficiency. This is typically done by minimizing the amount of substance that is wasted in non-targeted interactions. Gold in bulk is highly inert and therefore difficult to incorporate in a defined chemical approach, whereas the ionic form is a well studied agent in RedOx-reactions (i.e.the method that was chosen in this work).  The increase in efficiency can be achieved using a spatially selective approach with gold salt, where only the fiber is decorated with gold.

Gold(III) chloride trihydrate was therefore the reagent of choice.

\subsection{Reduction agent}


The reaction we wanted to happen, is the reduction of gold. Gold in bulk electrically conductive and promises better properties for usage.

Redox-Reaction (Two main ingredients, reduction and oxidation agent)

Draw redox-equation

The criteria for our reduction agent. Not to aggressive, because we want to limit reaction to gold. Polyurethane cannot be modified. should be able to reduce gold. 
cheap and easily accessible

optimally biocompatible


