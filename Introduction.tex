\section{Introduction}
Biomedical Engineering, with all its different forms, ranging from Biomechanics over Molecular Bioengineering to Bioelectronics, has experienced a surge in interest during the last few decades and holds the potential to change almost every aspect of the applied healthcare sector.\cite{BiomedicalIndustry} Recently established technologies like optogenetics,\cite{Optogenetics} where in vivo protein transcription on a cellular level is modified upon photonic stimulation, or new and exciting technologies like Medical AI\cite{MedicalAI} hold the potential to truly disrupt the industry. But biomedical engineering can also improve existing solutions. Park et al.\cite{EpicardialMesh} for example succeeded in designing an epicardial mesh made of electrically conductive and elastic material to resemble the innate cardiac tissue and confer cardiac conduction system function. This could reframe our perspective on treatment of cardiac malfunction. We live in very exciting times, where new fields emerge regularly, some more promising than others. One of the promising ones are stretchable electronics.\\
In recent years, stretchable electronics have gained a tremendous amount of attention. \cite{Cherenack, Lee} A lot of research has been done on how to design stretchable electronics devices, addressing different needs, ranging from wearable to implantable bioelectronic devices. Studies that parse the possible future applications describe examples like mapping lesions of cardiac tissue in real-time\cite{DiBiase}, development of a human-machine-interface\cite{Hochberg} or skin-based status monitors\cite{Kim}. Current methods however, fail to satisfy high conductivity, mechanical stability, non-planar design possibility and biocompatibility at the same time. Lee et al.  showed a facile and cost-effective approach to production of highly strain-sensitive fibres that comply with most of the before mentioned properties.\cite{Lee} Yet, the long-term biocompatibility remains to be determined. For successful realisation of said concepts, we need a highly stretchable, biocompatible, conductive fiber, able to reflect the non-planar and complex surface of the site of application.\\
In this work, making use of electroless and mild incorporation of solid gold nanoparticles in a non-conductive material, which is widely used in biomedical applications \cite{Pinchuk}, we succeeded in designing an simple protocol to fabricate  highly stretchable fiber with good strain sensing capabilities in ambient conditions.\\