\section{Introduction}
Biomedical Engineering, with all its different forms, ranging from Biomechanics over Molecular Bioengineering to Bioelectronics, has experienced a surge in interest during the last few years and is about to transform almost any aspect in the healthcare sector. \cite{BiomedicalIndustry} Recently established technologies like optogenetics \cite{Optogenetics}, where protein transcription of an arbitrarily complex organism is modified upon photonic stimulation, or new and exciting technologies like Medical AI \cite{MedicalAI} hold the potential to truly disrupt the industry. But biomedical engineering can also improve existing solutions. Park et al. \cite{EpicardialMesh} e.g. succeeded in designing an epicardial mesh made of eletrically conductive and elastic material to resemble the innate cardiac tissue and confer cardiac conduction system function. This could reframe our perspective on treatment of cardiac malfunction and lead to enhanced patient wellbeing. Currently we live in a very exciting time, where new fields emerge regularly, some more promising than others. One of the promising ones are stretchable electronics.\\
In recent years, stretchable electronics have gained a tremendous amount of attention. \cite{Cherenack, Lee} A lot of reasearch has been done on how to design stretchable electronics devices, adressing different needs, ranging from wearable to implantable bioelectronic devices. Studies that parse the possible future applications include examples like mapping lesions of cardiac tissue in real-time, development of a human-machine-interface or skin-based status monitors. \cite{Hochberg,Kim} Current methods however, fail to satisfy high conductivity, mechanical stability, non-planar design possibility and biocompatibility at the same time. Lee et al. \cite{Lee} showed a facile and cost-effective approach to production of highly strain-sensitive fibers that comply with most of the beforementioned properties. Yet, the long-term biocompatibility remains to be determined. For successful realisation of said concepts, we need a highly stretchable, biocompatible, conductive fiber able to reflect the nonplanar surface of the site of application.\\
In this work, making use of electroless coating of a non-conductive material, which is widely used in biomedical applications \cite{Pinchuk}, we succeeded in designing an simple protocol to fabricate  highly stretchable fiber with good strain sensing capabilities in ambient conditions.\\