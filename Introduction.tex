\section{Introduction}
Biomedical Engineering, with all its different forms, ranging from Imaging to Molecular and Bioelectronics, has experienced a surge in interest during the last few years and transforms the whole healthcare sector. \cite{BiomedicalIndustry} Recently established technologies like optogenetics \cite{Optogenetics}, where protein transcription is modified upon photonic stimulation, or exciting technologies like Medical AI \cite{MedicalAI} hold the potential to truly disrupt the industry. But biomedical engineering can also improve existing solutions. Park et al. \cite{EpicardialMesh} succeeded in designing an epicardial mesh made of eletrically conductive and elastic material to resemble the innate cardiac tissue and confer cardiac conduction system function. This could reframe our perspective on treatment of cardiac malfunction and lead to better patient outcome. But  

In recent years, stretchable electronics have gained a tremendous amount of attention. \cite{Cherenack, Lee} A lot of reasearch has been done on how to design stretchable electronics devices, adressing different needs, ranging from wearable to implantable bioelectronic devices. Studies that parse the possible future applications of stretchable conductive bioelectronic devices describe possible applications. Mapping lesions of cardiac tissue in real-time, development of a human-machine-interface or skin-based status monitors are just some to name a few. \cite{Hochberg,Kim} Current methods however, fail to satisfy high conductivity, mechanical stability, non-planar orientation and biocompatibility at the same time. Lee et al. \cite{Lee} showed a facile and cost-effective approach to production of highly sensitive fibers. Yet, the long-term biocompatibility remains to be determined. For successful realisation of said concepts, we need a highly stretchable, biocompatible, conductive fiber that reflect the nonplanar curvilinear surface of the site of application.

In this work, making use of electroless coating of a non-conductive material, which is widely used in biomedical applications \cite{Pinchuk}, we succeeded in designing an simple protocol to fabricate  highly sensitive fiber with good strain sensing capabilities in ambient conditions.
Advances in Biomedical Applications have great value... Innovate medicine and allow better therapy, faster diagnostic and more authentic implants. Biosensors have reached to measure biological and chemical sensors. Mechanical modalities underepresented. With mechanobiology we have field, where influence is shown. Stem Cells, Cancerous Cells, Tissue adapt and form according to mechanical influence. High-energy tissue (tendon/ligaments) largely black box, many things still open. We need strain sensing. Current methods have drawbacks. Our stuff, best stuff. 

There's a need for yadayada.

In this work...