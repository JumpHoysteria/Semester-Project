\hfill \newpage
\section{Outlook}
\label{sec:Outlook}

Our outlook on the described technology is positive. Using Ascorbic acid, the water-soluble Vitamin C, as a reducing agent, we have been successful in producing fibres that show linear strain-resistance behaviour in a physiological range of 20\%. Yet, there is a high variability in exhibited strain-resistance behaviour, which might hinder the direct integration in a more complex device. However, upon identification of the governing factors, we can minimise variability and increase reproducibility of previously mentioned behaviour.\\
The suggestions in advancement are two-fold and can be divided into either further optimization of the already described protocol parameters or by modification of the fiber production protocol.\\
When optimising the existing protocol, it becomes apparent that further inspection is needed. In order to gain insight and describe better the dynamic of the conducting gold particles during stretching, we suggest to image the fibres upon strain by SEM. Then with new information, we can suggest further advancements accordingly. One possible outcome could be that upon stretching the previously homogeneously distributed gold nanoparticles divide into a few big compartments which are interspaced by non-conducting areas, ultimately failing conduction. Since this could be a sign of missing or heterogenous adhesion on the fibre surface, we could propose to modify fibre  surface roughness. Furthermore, we also see potential in furhter standardizing both resistance measurement protocols (Preparation/Measurement).\\
When modifying the fibre production, we could propose to enclose the fiber in an elastomeric layer to allow for increased stability and sustained integrity of the conducting network. Another suggestion could be to introduce an in-situ reduction process similar to work done by Wei, where by fully immersing the non-conducting polymer in the reaction space, Wei's group was able to increase conductivity by 3-5 fold, when comparing it to conventional approaches.\cite{Wei}





